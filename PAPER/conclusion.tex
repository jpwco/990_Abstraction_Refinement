\section{Conclusion}
\label{sec:conclusion}

This work presents a novel technique for reasoning about movement \emph{ex post facto}.
We coarsen $x,y,z$ position data to a unit cube space akin to occupancy grids.
Within this space, we identify properties of movement for individuals and groups, as well as properties of paths.
We implement this technique in a tool, and show examples of the tool's output as it identifies properties from real world position trace files.
This work demonstrates the potential of this kind of spatial abstraction for finding useful properties of moving things.

In future work, we would like to explore properties of a host of actors (swarms), apply more efficient representations of cube space such as OctoMaps, and explore the extension of these techniques to software memory shapes.


