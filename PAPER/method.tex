\section{Method}
\label{sec:method}

\begin{tabular}{| p{2.8cm} | p{11.5cm} | }
\hline
PROPERTY & FORMALISM \\ \hline
Teleports &  $\phi_{\{teleport\}} = \exists  p_t, p_{t+1} | \lnot Adjacent(b_{p_t}, b_{p_{t+1}})$ \\ \hline
Hot Box & $\phi_{\{hotBoxOfSizeN\}} = \exists b_i, v_i \in E | n \geq |\{e \in E | e(v_x, v_i), v_x \in E\}$ \\ \hline
\end{tabular}

Method:
\begin{itemize}
 \item partition space into boxes
 \item enumerate the boxes
 \item map a concrete positional trace through space onto its uniquely corresponding boxes
 \item sync time frames of box-traces to infer spatio-temporal properties
 \item using the spatio-temporal box-trace, derive properties (given elsewhere)
\end{itemize}

\emph{Selecting appropriate box size}

Different box sizes may generate very different properties.

\emph{Too small of a box} If you choose the boxes to be relatively small, the trace may show a teleport, which we define as a spatially disconnected box-trace.
If there are gaps of disconnected boxes, and two entities collide in the unaccounted for space, the violation of box-independence will not be inferred.

\emph{Too big of a box} If you choose the boxes to be relatively large, this may be too rough of an over approximation.
For an extreme example, an entity never registers "leaving home" if its initial box is the size of the given dimensions.
Sometimes large over-approximations are desired, such as if you only want to examine the movements across two ``hemispheres" of an entity's space.
