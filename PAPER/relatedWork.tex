\section{Related Work}
\label{sec:related}


Work in using software execution trace data to infer invariants inspires our work, based on the Ph.D thesis of Ernst~\cite{ernst2000dynamically}.
Ernst's thesis is implemented in the Daikon~\cite{ernst2007daikon}~tool, that takes as input a trace of variable values and then outputs properties ("invariants") about the values seen in that trace.
Likewise, our method takes as input spatial traces and outputs properties observed during those traces.
Unlike Daikon, which first instruments source code to generate traces, our method assumes that spatial traces are already generated by some logging mechanism of the system, in our case, ROS, the Robot Operating System~\cite{quigley2009ros}.
Daikon has previously been applied to other domains such as hardware error detection~\cite{sahoo2008using}.

In the robotics domain, using unit cubes to reason about spaces dates back at least as far as Moravec and Elfes's occupancy grids~\cite{moravec1985high}.
This work used a unit space to map obstacles detected by sonar and is the root of cannonical approaches to planning and obstacle avoidance~\cite{siegwart2011introduction,elfes1989using,borenstein1989real}.
We likewise partition spaces into coarser, regular volumes with simple occupancy properties.
Occupancy grids have recently been refined by the efficient OctoMap~\cite{wurm2010octomap}, who's efficiency we hope to leverage in future work in large spaces and multiple scales.
Unlike all known previous work with unit cubes, used for mapping and planning, we reason about abstract properties of past movements.

We chose unit cubes because they are easy to work with and span \rthree. 
In 1885, Fedorov~\cite{fedorov1885elements} showed that there are exactly five ways to tessellate \rthree, and we examined the other shapes but decided that none offers advantages outweighing the ease of working with the unit cube.

Blank~\etal~ use video to trace human `space-time shapes' and then identify actions based on the shapes~\cite{blank2005actions}.
We likewise consider the space occupied by an object as is moves through space, but whereas Blank~\etal~ examine a highly-detailed image with individual moving parts, and the speeds of those parts, we consider an object to be a point in space occupying a box.

Another approach to spatial abstraction, not involving cube spaces, is explained by Shahar and Molina~\cite{shahar1998knowledge}, who studied the properties of car traffic flow in a two-dimensional plane.
They were interested in examining the properties of groups of cars in congestion by modeling the transition from fluid-like behavior to solid behavior.
Like their work, we are coarsening space and examining properties of objects abstractly.
Unlike their approach we work in unit space and search for properties of both individual actors as well as groups.

Gatalsky~\etal~applied space-time-cube visualizations to map a trace of movement over time to a three-dimensional cube~\cite{gatalsky2004interactive}.
Like this work we are tracing movement over time, but unlike this work we reasoning about abstract properties and our main purpose is not visualization.


