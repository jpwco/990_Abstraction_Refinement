\section{Background}
\label{sec:background}

Many people conceive of space in the manner of Descartes, where a fixed world frame with three perpendicular axes identifies an exact position at any point.  
Yet more sophisticated observers follow Minkowski~\cite{minkowski1952space} and imagine curvature in space time.
We're interested in a coarsening of these perspectives, imaging the universe cut into regular stacks of identical boxes, akin to an occupancy grid~\cite{moravec1985high}, or to use more technical terms, a tessellation of \rthree by a unit cuboid.

%For example, given a world frame centered directly under your feet, a point in your thumb might be described as $p_{thumb} = (x,y,z) = (0.2m, 0.3m, 0.4m)$.
%As your thumb moves, the Cartesian coordinates describing this point take on different values over time.
%If we consider some interval, $\delta t$, the set of all positions of $p_{thumb} = P$ describes the movement of this point during $\delta t$.

%However, with real systems in the world there are at least two limitations to $P$: $P$ is a finite sequence of points, and any $p \in P$ is uncertain with a quantifiable confidence.
%Let's denote $P'$ as a finite, uncertain version of $P$.
%Because $P'$ is uncertain, we can over-approximate the set of position points of our thumb by considering a kind of box, a bounding cubic volume defined by a set of 8 points, $v = {a,b,c,d,e,f,g,h}$.
%This over-approximation is useful because we can then imagine the box moving in space and if the box never intersects with an obstacle during a movement, they we can move our thumb as we imagined and know that we won't have any collisions.

This kind of coarsening is used in robot planning to reason about collision-free trajectories, specifically, planning into the future~\cite{siegwart2011introduction}.

In this paper, we're considering a similar over-approximation of the position of some \emph{actor} within space. 
The key difference is that we consider properties of spatial abstractions that happened in the past, instead of the future.
To do this, we adopt a very successful methodology from software engineering and implemented in the tool Daikon~\cite{kataoka2001automated}.
Daikon uses a record of what happened during the execution of a program, called a \emph{program trace}, to analyze high-level properties of software behavior.
We consider a \emph{spatial trace}, a finite time sequence of positions, and try to reason about higher-level properties of the movement.

%If we could record, or write down, a sequence of point 
%Increasingly, our world is full of things that know where they are in space.
%Some of these things can move in space and we created them and they move based on rules we design.

