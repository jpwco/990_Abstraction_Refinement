\section{Background}
\label{sec:background}

Things exists in spatial and temporal relation to one another.
Many people conceive of space following in the manner of Descartes, where there is a fixed world frame and objects exists at a point in $R^3$.
For example, given a world frame centered directly under your feet, a point in your thumb might be described as $p_{thumb} = (x,y,z) = (0.2m, 0.3m, 0.4m)$.
As your thumb moves, the Cartesian coordinates describing this point take on different values over time.
If we consider some interval, $\delta t$, the set of all positions of $p_{thumb} = P$ describes the movement of this point during $\delta t$.

However, with real systems in the world there are at least two limitations to $P$: $P$ is a finite sequence of points, and any $p \in P$ is uncertain with a quantifiable confidence.
Let's denote $P'$ as a finite, uncertain version of $P$.
Because $P'$ is uncertain, we can over-approximate the set of position points of our thumb by considering a kind of box, a bounding cubic volume defined by a set of 8 points, $v = {a,b,c,d,e,f,g,h}$.
This over-approximation is useful because we can then imagine the box moving in space and if the box never intersects with an obstacle during a movement, they we can move our thumb as we imagined and know that we won't have any collisions.

This kind of over-approximation is used in robot planning to reason about collision-free trajectories, specifically, planning into the future.

In this paper, we're considering a similar over-approximation, or abstraction, of the position of some \emph{actor}, the thing that moves, within space, 
The key difference is that we consider properties of spatial abstractions that happened in the past, instead of the future.
To do this, we adopt a very successful methodology from software engineering and implemented in the tool Daikon~\cite{kataoka2001automated}.
Daikon uses a record of what happened during the execution of a program, called a \emph{program trace}, to analyze high-level properties of software behavior.
We consider a \emph{spatial trace}, a finite time sequence of positions, and try to reason about higher-level properties of the movement.

If we could record, or write down, a sequence of point 
Increasingly, our world is full of things that know where they are in space.
Some of these things can move in space and we created them and they move based on rules we design.

