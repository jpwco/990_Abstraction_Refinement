\documentclass[12pt]{article}

\begin{document}

\begin{table}[h]
\begin{tabular}{|l|l|}
\hline
Temp. PO &  0\\ \hline
Spatial PO & 0  \\ \hline
Spatial-Temp PO & 0 \\ \hline
Points of Interest & 0 \\ \hline
\end{tabular}
\end{table}

{\sl Catching Objects in Flight}
\cite{kim2014catching}
\begin{itemize}
  \item Shape of trace
  \item Point of interest
  \item Optimize at each time slice
\end{itemize}

{\sl Concentric Tube Robot Design and Optimization
Based on Task and Anatomical Constraints}
\cite{bergelesconcentric}
\begin{itemize}
  \item 3D curves
  \item Total spatial order
  \item Velocity (kinematics)
\end{itemize}

{\sl A General, Fast, and Robust Implementation of the
Time-Optimal Path Parameterization Algorithm}
\cite{pham2014general}
\begin{itemize}
  \item Velocity/Acceleration parameterization
  \item Constrained path (similar to above)
\end{itemize}

{\sl Learning Grounded Finite-State Representations
from Unstructured Demonstrations}
\cite{niekum2014learning}
\begin{itemize}
  \item Segment using hidden Markov model
  \item Get ``best" coordinate frames by clustering
        and dropping the singletons
  \item Discover ``dynamic movement primitives"
  \item Get closest matching pattern to runs of
        demonstrations with clear start and end points
\end{itemize}

{\sl Communication-aware Information Gathering with
Dynamic Information Flow}
\cite{kassir2014communication}
\begin{itemize}
  \item Reformulation of network flow
  \item Experiments use spatial containment ideas
\end{itemize}

{\sl BRVO: Predicting Pedestrian Trajectories Using
Velocity-Space Reasoning}
\cite{kim2014brvo}
\begin{itemize}
  \item Velocity/Acceleration
  \item Collision avoidance
  \item Uses 6D vectors: 2D pos, 2D vel, 2D ``preferred" vel
\end{itemize}

{\sl Multi-agent Plan Reconfiguration under Local
LTL Specifications}
\cite{guo2014multi}
\begin{itemize}
  \item Containment spheres
  \item Point-of-interest (with given starting point)
  \item Temporal P.O.
\end{itemize}

{\sl Distributed Data Fusion for Multirobot Search}
\cite{hollingerdistributed}
\begin{itemize}
  \item Point of interest
  \item Temporal P.O. (to determine how to ``fuse" data)
  \item Containment (detecting targets within some 
                     discrete cell)
\end{itemize}

\bibliographystyle{plain}
\bibliography{properties}

\end{document}
