\section{Expected Outcome} 

\emph{Connections with analyses studied in class}

As we hope to infer spatial models from trace data over multiple runs, our proposed tool shares many similarities with Daikon, which infers program invariants from (possibly) multiple runs given some input.
Like Daikon, we hope to work backwards from the trace data; because of this, like Daikon, the quality/correctness of the models we extract will be closely tied to the quality/robustness of the input data set.
We have also taken the idea of confidence thresholds from Daikon, to determine when a spatial event occurs with regularity, and when it just happened to occur a limited number of times.
Unlike Daikon, we are not looking at invariants, so our models can be much more lax; if a spatial property is missing from some traces, but is present in most, we will still include this property in the model.
%The publisher/subscriber architecture that captures events during a robot's task is similar to instrumenting your code to output desired variable values (Daikon).
%However, raw sensor data is different than instrumented code because these sensors will output periodically, regardless of whether values have changed, and this creates a lot of data noise which must be filtered.
%The way we will extract models from multiple BAG'' files (containing spatial events) may employ methods of time-series (state-series in our case) analysis and multivariate analysis which are used in Jenkins.   
We hope to use the spatial models to identify behavior around areas where runtime failures occur, though this will not be an automatic recovery, as in Carzaniga~\etal~\cite{carzaniga2013automatic}.

\emph{Advances in state of practice/art}

If our tool can produce meaningful spatial models, this study would be an advance in the state of the art, as there is currently no work being done in spatial model extraction.
We hope this advance in the state of the art may be put into practice on general robotics systems. 
To do this, our tool may need to be extensible, so that users may easily define spatial events that would be important to their specific system.
